\section{Fairness Metrics}

The following component \textbf{C4} illustrate four fairness metrics 
\footnote{P. Garg, J. Villasenor, and V. Foggo. Fairness metrics: A comparative analysis. In IEEE International Conference on Big Data (Big Data), pp. 3662–3666. IEEE, 2020.} 
for quantitively measuring the level of bias in the model prediction: statistical parity difference (SPD) , disparate impact (DI), equal opportunity differences (EOD), and average odds difference (AOD).

\begin{visualComponent}
    \name{FairMetrics}
    \metrics{SPD, DI, EOD, AOD}
    \interaction{True}
    \data{Original}
\end{visualComponent}

\begin{boxK}
    By default, the \textbf{Original} prediction is derived from a logistic regression model. The model is trained and tested using a 68\% split ratio on the available data, incorporating all features.
\end{boxK}

\par The two confusion matrices display the model predictions, which are consistent with the output panel of \textbf{C3}. 
The four panels below visualize the fairness metrics calculated from the confusion matrices. For each metric, solid vertical lines indicate its current value and dotted vertical lines represent its fair/baseline value. 
Values that are biased against the unprivileged group are shown in grey. The distance between the current and the baseline value indicates the bias level of the model predictions. 
When users click a metric panel, its mathematical formula will be displayed in the middle panel, and the confusion matrices are highlighted accordingly.
When users modify the values confusion matrices then press the "enter" key, the fairness metrics get updated accordingly.
