\section{Debiasing Methods}

\par In this chapter, we will demonstrate bias mitigation methods through our components, including three pre‐processing methods, two in-processing methods, and one post-processing methods.
 
\subsection{Fairness Pre-Processing Methods}
\par Pre-processing methods aim to mitigate the biases in the training data. 
We present three pre-processing methods in this section. We use component C5 to showcase the pre-processing methods. 
Additionally, we integrate C5 with C3 and C4 to facilitate a comparison of fairness metric values in the model predictions before and after addressing biases in the training data.
Examples are shown below. 

\subsubsection{Reweighing}
Refer to 
\footnote{F. Kamiran and T. Calders. Data preprocessing techniques for classification without discrimination. Knowledge and information systems, 33(1):1–33, 2012.}
for a detailed explanation of the \emph{Reweighing} algorithm.

\begin{enumerate}
    \item Upon clicking the “Reweighing” button, the modified data and the weights (only applicable for Reweighing) applied to each subgroup will be displayed in the right panel.
    \item Subsequently, C3 will appear, with the debiased training data replacing the original training data. Users can then proceed to re-train and re-evaluate the model to generate new predictions.
    \item Once the new predictions are generated, C4 will appear, enabling a comparison of fairness metric values on the prediction before and after applying the pre-processing method.
\end{enumerate}

\begin{VCSet}
    \begin{visualComponent}
        \name{PreProcess}
        \type{Reweighing}
    \end{visualComponent}
    
    \begin{visualComponent}
        \name{MLPipeline}
        \trainData{Reweighing}
        \model{LR}
    \end{visualComponent}

    \begin{visualComponent}
        \name{FairMetrics}
        \metrics{SPD, DI, EOD, AOD}
        \interaction{True}
        \data{Reweighing}
    \end{visualComponent}
\end{VCSet}


\subsubsection{Learning Fair Representations}

Refer to 
\footnote{R. Zemel, Y. Wu, K. Swersky, T. Pitassi, and C. Dwork. Learning fair representations. In International conference on machine learning, pp.325–333, 2013.} 
for a detailed explanation of the \emph{Learning Fair Representations} algorithm.
See the \emph{Reweighing} section for the use of the following components.

\begin{VCSet}
    \begin{visualComponent}
        \name{PreProcess}
        \type{LFR}
    \end{visualComponent}
    
    \begin{visualComponent}
        \name{MLPipeline}
        \trainData{LFR}
        \model{LR}
    \end{visualComponent}
    
    \begin{visualComponent}
        \name{FairMetrics}
        \metrics{SPD, DI, EOD, AOD}
        \interaction{False}
        \data{LFR}
    \end{visualComponent}
\end{VCSet}


\subsubsection{Optimized Pre-processing}
Refer to 
\footnote{F. Calmon, D. Wei, B. Vinzamuri, K. Natesan Ramamurthy, and K. R. Varshney. Optimized pre-processing for discrimination prevention. Advances in neural information processing systems, 30, 2017.} 
for a detailed explanation of the \emph{optimized pre-processing} algorithm.
See the \emph{Reweighing} section for the use of the following components.

\begin{VCSet}
    \begin{visualComponent}
        \name{PreProcess}
        \type{OptimPreproc}
    \end{visualComponent}
    
    \begin{visualComponent}
        \name{MLPipeline}
        \trainData{OptimPreproc}
        \model{LR}
    \end{visualComponent}
    
    \begin{visualComponent}
        \name{FairMetrics}
        \metrics{SPD, DI, EOD, AOD}
        \interaction{False}
        \data{OptimPreproc}
    \end{visualComponent}
\end{VCSet}

\subsection{In-processing methods}

In-processing methods use ML models that take fairness into account, typically, by adding a fairness term when optimizing the model. 
We present two in‐processing methods in this section. 
We reuse component \textbf{C3} to showcase the in‐processing methods, and we integrate \textbf{C3} with \textbf{C4} to facilitate a comparison of fairness metric values before and after applying in‐processing methods. 
Examples are shown below.


\subsubsection{Prejudice Remover}

Refer to 
\footnote{T. Kamishima, S. Akaho, H. Asoh, and J. Sakuma. Fairness-aware classifier with prejudice remover regularizer. In Joint European conference on machine learning and knowledge discovery in databases, pp.35–50. Springer, 2012.} 
for a detailed explanation of the \emph{Prejudice Remover} algorithm.

\begin{enumerate}
    \item Users can train and evaluate the fairness model to generate predictions.
    \item \textbf{C4} will appear Subsequently, enabling a comparison of fairness metric values on the prediction before and after applying the in-processing method.
\end{enumerate}

\begin{VCSet}
    \begin{visualComponent}
        \name{MLPipeline}
        \trainData{Original}
        \model{PrejudiceRmv}
    \end{visualComponent}
    
    \begin{visualComponent}
        \name{FairMetrics}
        \metrics{SPD, DI, EOD, AOD}
        \interaction{False}
        \data{PrejudiceRmv}
    \end{visualComponent}
\end{VCSet}

\subsubsection{Adversarial Debiasing}
Refer to 
\footnote{B. H. Zhang, B. Lemoine, and M. Mitchell. Mitigating unwanted biases with adversarial learning. In Proceedings of the 2018 AAAI/ACM Conference on AI, Ethics, and Society, pp. 335–340, 2018.} 
for a detailed explanation of the \emph{Adversarial Debiasing} algorithm.
See the \emph{Prejudice Remover} section for the use of the following components.


\begin{VCSet}
    \begin{visualComponent}
        \name{MLPipeline}
        \trainData{Original}
        \model{Adversarial}
    \end{visualComponent}
    
    \begin{visualComponent}
        \name{FairMetrics}
        \metrics{SPD, DI, EOD, AOD}
        \interaction{False}
        \data{Adversarial}
    \end{visualComponent}
\end{VCSet}


\subsection{Post-processing methods}
\par Post-processing methods improve the fairness of predictions by directly changing the predicted results, for example, by modifying the decision threshold. 
In this section we show one post-processing method, \emph{Reject Option Classification}
\footnote{F. Kamiran, A. Karim, and X. Zhang. Decision theory for discrimination-aware classification. In 2012 IEEE 12th International Conference on Data Mining, pp. 924–929. IEEE, 2012.}
, through the component \textbf{C6}.

\begin{enumerate}
    \item Upon clicking the "ROC" button, the ajusted prediction will be displayed in the right panel.
    \item \textbf{C4} will appear Subsequently, enabling a comparison of fairness metric values on the prediction before and after applying the post-processing method.
\end{enumerate}


\begin{VCSet}
    \begin{visualComponent}
        \name{PostProcess}
        \type{ROC}
    \end{visualComponent}
    
    \begin{visualComponent}
        \name{FairMetrics}
        \metrics{SPD, DI, EOD, AOD}
        \interaction{False}
        \data{ROC}
    \end{visualComponent}
\end{VCSet}